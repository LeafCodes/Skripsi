\chapter{Kesimpulan dan Saran}

\section{Kesimpulan}

Berdasarkan penelitian yang telah dilakukan, dapat ditarik beberapa kesimpulan sebagai berikut:
\begin{enumerate}
    \item Penggunaan arsitektur dengan 1 hidden layer menunjukkan bahwa semakin banyak neuron pada hidden layer semakin sedikit iterasi yang diperlukan sampai pada 7 hidden neuron dan meningkat kembali.
    \item Penggunaan arsitektur dengan jumlah neuron yang sama pada tiap layer menunjukkan semakin banyak hidden layer semakin lama pula waktu iterasi yang dibutuhkan. Hal ini terjadi pada konfigurasi 3, 4, 5, dan 6 hidden layer.
    \item Penggunaan arsitektur dengan jumlah bobot yang sama menunjukkan 4 hidden layer selalu memiliki waktu komputasi yang lebih lama dibandingkan dengan jumlah hidden layer yang lebih sedikit. Satu hidden layer selalu memiliki waktu komputasi yang lebih sedikit dibanding dengan jumlah hidden layer lebih banyak. 
    \item Penggunaan arsitektur dengan jumlah hidden neuron yang lebih sedikit dari neuron pada layer masukan selalu gagal mencapai 100\%. Hal ini berlaku pada jumlah hidden neuron pada hidden layer ke berapapun.
\end{enumerate}

\section{Saran}

Dari penelitian yang dilakukan, masih terdapat banyak kekurangan. Penelitian ini hanya berfokus pada aristektur Multilayer Perceptron tanpa fungsi aktifasi pada lapisan tersembunyi dan dilakukan pada data yang dapat dipisahkan secara linear.   

\begin{enumerate}
    \item Dapat menggunakan jaringan dengan arsitektur yang lebih beragam seperti Convolutiona Neural Network, Reccurent Neural Network, dll.

    \item Melakukan penelitian dengan data yang lebih beragam.Misalnya saja dengan data yang tidak dapat dipisahkan secara linear atau dengan data yang memiliki outlier yang bervariasi


\end{enumerate}
