% \chapter*{LAMPIRAN}
% %\addcontentsline{toc}{chapter}{LAMPIRAN}	

% \section{Isi Lampiran}

% Lampiran bersifat opsional bergantung hasil kesepakatan dengan pembimbing 
% dapat berupa:

% \begin{enumerate}
% \item Bukti pelaksanaan Kuesioner seperti pertanyaan kuesioner, resume jawaban 
% responden, dan dokumentasi kuesioner. 
% \item Spesifikasi Aplikasi atau Sistem yang dikembangkan meliputi spesifikasi 
% teknis aplikasi, tautan unduh aplikasi, manual penggunaan aplikasi, hingga 
% screenshot aplikasi. 
% \item Cuplikan kode yang sekiranya penting dan ditambahkan. 
% \item Tabel yang terlalu panjang yang masih diperlukan tetapi tidak 
% memungkinkan untuk ditayangkan di bagian utama skripsi.
% \item Gambar-gambar pendukung yang tidak terlalu penting untuk ditampilkan di 
% bagian utama. Akan tetapi, mendukung argumentasi/pengamatan/analisis.
% \item Penurunan rumus-rumus atau pembuktian suatu teorema yang terlalu 
% panjang dan terlalu teknis sehingga Anda berasumsi bahwa pembaca biasa 
% tidak akan menelaah lebih lanjut. Hal ini digunakan untuk memberikan 
% kesempatan bagi pembaca tingkat lanjut untuk melihat proses penurunan 
% rumus-rumus ini.
% \end{enumerate}