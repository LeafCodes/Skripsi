Puji dan syukur telah dilayangkan kepada kehadirat Allah SWT yang telah melimpahkan kasih dan sayang-Nya, sehingga tugas akhir yang berjudul "Pengaruh Jumlah \textit{Hidden Layer} Terhadap Kinerja
Akurasi dan Waktu Pembelajaran pada Jaringan Syaraf Tiruan" dapat terselesaikan. Adapun tujuan dari penyusunan tugas akhir ini adalah untuk memenuhi salah satu persyaratan dalam menempuh ujian jenjang sarjana pada program studi Teknik Elektro yang berada di Fakultas Teknik Elektro dan Informatika Universitas Gadjah Mada.

Dengan tersusunnya laporan ini, pemahaman diperoleh bahwa dalam proses penyusunannya tak terhindar dari bantuan, doa, dan bimbingan dari berbagai pihak kepada. Oleh karena itu, banyak terima kasih diungkapkan kepada:
\begin{enumerate}
    \item Bapak Ir. Hanung Adi Nugroho, S.T., M.Eng., Ph.D., IPM. selaku Ketua
	Departemen Teknik Elektro dan Teknologi Informasi dan Ir. Adha Imam Cahyadi,
	S.T., M.Eng., D.Eng., IPM. selaku Ketua Program Studi S1 Teknik Elektro
	Fakultas Teknik Universitas Gadjah Mada.
	
    \item Ir. Lesnanto Multa Putranto, S.T., M.Eng, Ph.D., IPM. selaku Sekretaris
	Departemen Teknik Elektro dan Teknologi Informasi.

    \item Bapak Dr. Ir. Risanuri Hidayat, M.Sc. selaku dosen pembimbing 1.
    \item Bapak Prof. Dr. Ir. Sasongko Pramono Hadi, DEA. selaku dosen pembimbing 2.
    \item Seluruh dosen, staf dan karyawan Program studi Teknik 
          Elektro Universitas Gadjah Mada. 
    \item Terkhusus kepada yang tercinta dan saya banggakan bapak Adri Hartono S.Si dan Sugihartini S.Pd yang telah banyak berkorban dalam mengasuh, mendidik, mendukung dan mendoakan penulis dengan penuh kasih sayang yang tulus dan ikhlas. Serta Risda Putri Indriani dan Fauzan Dwiputra yang turut mendukung penulis secara moral.
    \item Sahabat-sahabatku dan rekan-rekan seperjuangan mahasiswa DTETI angkatan 2019.
\end{enumerate}


Akhir kata penulis berharap semoga skripsi ini dapat memberikan manfaat bagi kita semua, aamiin.