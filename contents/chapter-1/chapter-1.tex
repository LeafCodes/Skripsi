\chapter{Pendahuluan}

\section{Latar Belakang}

Kecerdasan artifisial (KA) atau, \textit{artificial intelligence} (AI) adalah bidang ilmu yang berfokus pada pengembangan sistem komputer untuk melakukan tugas-tugas yang membutuhkan kecerdasan manusia, seperti belajar, 
menalar, dan mengenali pola\cite{rich_Knight_ai}. Bidang ini belakangan berkembang sangat pesat dan menjadi perhatian global. Data dari laporan terbaru menunjukkan 
bahwa investasi global dalam teknologi KA terus meningkat secara signifikan, dengan perkiraan mencapai lebih dari 300 miliar dolar Amerika pada tahun 2023\cite{300mn}. 
Teknologi KA telah diintegrasikan secara luas ke dalam berbagai sektor.
Dalam bidang kesehatan, KA digunakan untuk menganalisis data medis dan membantu diagnosis penyakit\cite{Early_Diagnosis_of_Alzheimer's_Disease}.
Sementara itu, dalam sektor keuangan, KA dimanfaatkan untuk mengoptimalkan portofolio investasi dan memprediksi kondisi obligasi di masa depan. 
Sejumlah platform digital besar, seperti Facebook dan Google, telah mengadopsi teknologi kecerdasan artifisial guna meningkatkan efisiensi layanan dan meningkatkan pengalaman penggunanya\cite{Kelleher2019-cj}.

Indonesia, melalui Badan Pengkajian dan Penerapan Teknologi (BPPT), menanggapi perkembangan ini dengan membentuk Strategi Nasional Kecerdasan Artifisial Indonesia 2020 
- 2045 (STRANAS KA)\cite{Stranaska}. Pembentukan STRANAS KA ini merupakan perwujudan Visi Indonesia Emas 2045 dan menekankan pentingnya penelitian dan penggunaan KA
untuk sektor-sektor yang diprioritaskan. Dalam bidang ilmu ini, salah satu teknik yang fundamental dan banyak dikembangkan adalah Jaringan Syaraf Tiruan (JST). 
Oleh karena itu, penelitian yang mendukung pengembangan JST, seperti yang dilakukan dalam skripsi ini, selaras dengan tujuan yang ditetapkan oleh pemerintah.

Jaringan syaraf tiruan telah menjadi fondasi bagi berbagai aplikasi KA, seperti pengenalan wajah, penerjemah bahasa, dan \textit{driver assistance system}\cite{Auto_driving}. 
Keberhasilan jaringan syaraf tiruan dalam menangani tugas-tugas kompleks didukung dengan meningkatnya daya komputasi dan ketersediaan data dalam jumlah besar 
melalui teknologi \textit{cloud}. 

Jaringan syaraf tiruan adalah model matematis yang meniru cara kerja otak manusia. Jaringan syaraf tiruan terdiri dari serangkaian \textit{neuron} yang terhubung satu sama lain dan tersusun membentuk lapisan-lapisan. 
\textit{Input layer} menerima data, \textit{output layer} digunakan untuk mengeluarkan prediksi, dan di antara keduanya terdapat satu atau lebih \textit{hidden layer}. 
\textit{Hidden layer} (HL) inilah yang bertanggung jawab untuk mengekstraksi dan mempelajari fitur-fitur kompleks dari data. Jumlah dan konfigurasi \textit{hidden layer} 
akan mempengaruhi kemampuan jaringan dalam memahami dan memodelkan pola dalam data. Dengan konfigurasi \textit{hidden layer} yang tepat, jaringan syaraf tiruan dapat 
meningkatkan kinerja dan efisiensi dalam berbagai tugas klasifikasi maupun regresi.

Salah satu permasalahan yang sering dihadapi dalam penggunaan jaringan syaraf tiruan adalah penentuan jumlah \textit{hidden layer} yang akan digunakan. 
Penentuan jumlah dan ukuran \textit{hidden layer} ini tidak selalu dapat terlihat langsung didalam data. Pemilihan jumlah dan ukuran \textit{hidden layer} memerlukan pengaturan 
yang baik untuk mendapatkan hasil yang diinginkan. Secara umum, jika jumlah \textit{hidden layer} terlalu sedikit, jaringan mungkin gagal mempelajari representasi yang cukup 
kompleks dari data. Sebaliknya, jika jumlahnya terlalu banyak, dapat terjadi \textit{overfitting}, yaitu jaringan menjadi terlalu cocok dengan data latih dan kinerjanya menurun 
saat diuji dengan data baru. Oleh karena itu, menemukan jumlah \textit{hidden layer} yang tepat merupakan tantangan  dalam merancang 
jaringan syaraf tiruan yang efektif. 

Beberapa pendekatan, seperti penggunaan \textit{cross-validation} atau teknik optimasi seperti algoritma genetika, telah digunakan untuk membantu menentukan 
jumlah \textit{hidden layer} yang optimal dalam suatu jaringan syaraf tiruan. Meskipun kecerdasan artifisial merupakan bidang penelitian yang masih berkembang, penelitian 
atas permasalahan ini diharapkan akan memperkuat kemampuan jaringan syaraf tiruan untuk menghadapi berbagai tugas pemodelan dan prediksi dengan lebih akurat dan efisien.

\section{Rumusan Masalah}

Berdasarkan latar belakang yang telah diuraikan, permasalahan utama dalam penelitian ini adalah penentuan jumlah dan ukuran \textit{hidden layer} pada suatu jaringan 
syaraf tiruan belum memiliki panduan yang jelas. pengurangan ataupun penambahan pada \textit{hidden layer} memberikan efek yang kompleks dan tidak selalu linier terhadap 
kinerja model dan diperlukan penelitian lebih lanjut. Oleh karena itu, rumusan masalah disusun seperti berikut:

\begin{enumerate}
    \item Ketidakpastian jumlah \textit{hidden layer} menyebabkan kesulitan dalam merancang arsitektur JST yang efektif.
    \item Penelitian mengenai efek apa yang terjadi pada pengurangan dan penambahan \textit{hidden layer} pada jaringan syaraf tiruan masih 
    menunjukan hasil yang beragam dan perlu dikaji lebih lanjut.
    \item Penelitian mengenai pengaruh jumlah \textit{hidden layer} terhadap waktu komputasi dan akurasi menghasilkan perbedaan yang beragam.
\end{enumerate}

\section{Tujuan Penelitian}

Dalam penelitian ini terdapat beberapa tujuan yang ingin dicapai, yaitu:

\begin{enumerate}
    \item Menganalisis pengaruh penambahan atau pengurangan \textit{hidden layer} terhadap tingkat akurasi yang dihasilkan dalam tugas klasifikasi.
    \item Menganalisis pengaruh penambahan atau pengurangan \textit{hidden layer} terhadap lama waktu komputasi yang diperlukan selama proses pelatihan model.
    \item Menemukan konfigurasi jumlah \textit{hidden layer} yang optimal untuk dataset yang digunakan, dengan mempertimbangkan \textit{trade-off} antara akurasi yang tinggi dan waktu komputasi yang efisien.
\end{enumerate}

\section{Batasan Penelitian}

Untuk memfokuskan cakupan penelitian, maka ditetapkan batasan-batasan masalah sebagai berikut:

\begin{enumerate}
    \item Objek Penelitian :
    Objek yang diteliti pada penelitian ini adalah pengaruh variasi jumlah \textit{hidden layer} pada arsitektur \textit{Multilayer Perceptron} (MLP) terhadap kinerja model dalam melakukan klasifikasi pada data \textit{linearly separable}.

    \item Waktu dan tempat penelitian :
    Penelitian ini dimulai pada Januari 2023 hingga Januari 2025

    \item Variabel :
    Variabel Bebas dari penelitian ini adalah jumlah \textit{hidden layer} pada arsitektur MLP. 
	Variabel Terikat dari penelitian ini adalah akurasi klasifikasi dan waktu komputasi yang diperlukan selama proses pelatihan (\textit{training}) model.
    
    \item Populasi dan Sampel :
    Populasi dalam penelitian ini adalah seluruh kemungkinan dataset yang \textit{linearly separable}. Sampel yang digunakan adalah satu dataset sintetis dengan spesifikasi 16 instance, 2 fitur, dan 3 kelas.

    \item Data Penelitian :
    Data yang digunakan adalah data sintetis yang dibuat secara khusus agar bersifat \textit{linearly separable}. Data ini memiliki 16 instance, 2 fitur, dan tiga kelas target.

\end{enumerate}

\section{Manfaat Penelitian}

Manfaat yang diharapkan dari penelitian ini adalah:

\begin{enumerate}
	\item Akademisi, Hasil penelitian ini dapat menambah wawasan ilmu dalam bidang kecerdasan artifisial, khususnya mengenai pemahaman mendalam tentang pengaruh kedalaman JST pada MLP serta menjadi pijakan untuk penelitian lanjutan.
	\item Praktisi, Memberikan panduan dalam melakukan tuning arsitektur jaringan saraf tiruan yang lebih terarah, sehingga dapat menghemat waktu dan sumber daya komputasi dalam pengembangan model.
\end{enumerate}

\pagebreak

\section{Sistematika Penulisan}

\noindent \textbf{BAB I PENDAHULUAN}

Pada ini dijelaskan latar belakang, rumusan masalah, batasan, tujuan, manfaat, dan sistematika penulisan.

\noindent \textbf{BAB II : TINJAUAN PUSTAKA DAN LANDASAN TEORI}

Pada bab ini dijelaskan teori dan penelitian yang digunakan sebagai acuan dalam penelitian.

\noindent \textbf{BAB III : METODOLOGI PENELETIAN}

Pada bab ini berisi tentang metodologi penelitian yang terdiri atas alat dan bahan, langkah kerja, dan alur tahapan penelitian.

\noindent \textbf{BAB IV : ANALISIS DAN PEMBAHASAN}

Pada bab ini dijelaskan hasil dari penelitian dan pembahasan pengujianya.

\noindent \textbf{BAB V : KESIMPULAN DAN SARAN}

Pada bab ini ditulis kesimpulan penelitian dan saran untuk penelitian selanjutnya.