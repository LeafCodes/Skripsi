% \textit{
% 	Dashboard camera is one of the tools that facilitate handling when a traffic accident occurs. However, it cannot directly prevent traffic accidents because it can only provide an overview of the situation that occurs around the vehicle. In the case of an accident, the conditions that occur in the vehicle are also important. In connection with that, a dashcam is needed that can find out the condition of the vehicle. to be able to find out the condition of the vehicle, a monitoring system feature can be developed that can store data on the server. the monitoring system communicates with the electronic control unit (ECU) to obtain vehicle condition data. One way to be able to communicate with the ECU is generally to use the controller area network (CAN) communication protocol. Therefore, a dashcam can be developed that can store data on a server and communicate using the CAN protocol.
% }

% \textit{
% In this research, hardware prototype design and program development of a monitoring system for dashcams are carried out. In hardware development, a prototype is made using two ESP32 which functions as ESP32-Sender and ESP32-Receiver. The ESP32-Sender will be simulated as a vehicle that has 5 data to be sent, namely speed, RPM, temperature, latitude, and longitude. The ESP32-receiver will receive data from the ESP32-sender via CAN and will send the data to the MQTT server using a SIM modem. Furthermore, the program will be made using Arduino IDE so that it can send and receive data using CAN and can send data to the MQTT server using SIM modem. Furthermore, it will be seen whether the CAN communication protocol can function properly between the two ESP32, and whether the ESP32-receiver can send data to the MQTT server properly.
% }

% \textit{
% The results of this research show that the GPRS communication protocol can work well based on testing in several location conditions. The test has the results of a signal strength indicator from the prototype monitoring system to the MQTT server which is indicated by a good Received Signal Strength Indication (RSSI) value. In addition, the prototype monitoring system can communicate using the CAN protocol. This can be done by using a CAN controller that has been integrated with ESP32.
% }

% \textit{
% Dashboard camera (dashcam) is a camera system installed on a vehicle to record activities around the vehicle. Dashcam works by using a small camera which then stores the recording on local memory. Due to the large video data, there are still few dashcams that can be monitored directly on the server, so dashcams cannot be monitored in real time. For this reason, another way is needed to monitor the vehicle by sending data that can provide information about the vehicle. To be able to find out vehicle information, a monitoring system feature can be developed that can store data directly on the server. The monitoring system communicates with the Electronic Control Unit (ECU) to get data from the vehicle. One way to communicate with the ECU is generally to use the Controller Area Network (CAN) communication protocol. Therefore, a dashcam can be developed that can store data on the server and communicate using the CAN protocol in real time.
% }

\textit{}

% \textit{
% In this research, hardware simulation design and program development of a monitoring system for dashcams are carried out. In hardware development, a simulation is made using two ESP32s that function as ECU simulators and dashcam simulators. The ECU simulator will be simulated as a vehicle that has 5 data to be sent, namely speed, RPM, temperature, latitude, and longitude. The dashcam simulator will receive data from the ECU simulator via CAN and will send the data to the MQTT server using a SIM modem. Furthermore, the program will be made using Arduino IDE so that it can send and receive data using CAN and can send data to the MQTT server using SIM modem. Furthermore, it will be seen whether the CAN communication protocol can function properly between the two ESP32s, and whether the dashcam simulator can send data to the MQTT server properly.
% }

\textit{}

% \textit{
% The results of this research show that the GPRS connection on SIM800L can work well based on testing in several location conditions. The test has the results of signal strength indicators from the monitoring system simulation to the MQTT server which is indicated by a good Received Signal Strength Indication (RSSI) value. In addition, the simulated monitoring system can communicate with each other using the CAN protocol. Indicated by the results of sending and receiving data through the same CAN line. Simulation testing in a relatively long distance continuously also gives good results, where data transmission failure in continuous testing is 4.2\%.
% }

\textit{}

\noindent\textbf{Keywords} : Artificial Neural Network, Hidden Layer