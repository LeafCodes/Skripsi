% \textit{Dashboard camera} adalah salah satu alat yang mempermudah penanganan ketika terjadi kecelakaan lalu lintas. Akan tetapi tidak bisa mencegah secara langsung terjadinya kecelakaan lalu lintas karena hanya dapat memberikan gambaran situasi yang terjadi di sekitar kendaraan saja. Dalam kasus kecelakaan, kondisi yang terjadi pada kendaraan juga merupakan suatu hal yang penting. Sehubungan dengan itu dibutuhkan dashcam yang dapat mengetahui kondisi kendaraan. Untuk dapat mengetahui kondisi kendaraan tersebut dapat dikembangkan fitur sistem \textit{monitoring} yang dapat melakukan penyimpanan data pada \textit{server}. Sistem \textit{monitoring} tersebut berkomunikasi dengan \textit{Electronic Control Unit} (ECU) untuk mendapatkan data kondisi kendaraan. Salah satu cara untuk dapat berkomunikasi dengan ECU umumnya yaitu digunakan protokol komunikasi \textit{Controller Area Network} (CAN). Oleh karena itu dapat dikembangkan dashcam yang dapat menyimpan data pada \textit{server} dan berkomunikasi menggunakan protokol CAN.

% \textit{Dashboard camera} (dashcam) adalah sistem kamera yang dipasang pada kendaraan untuk merekam aktivitas disekitar kendaraan. Dashcam bekerja dengan menggunakan kamera kecil yang kemudian menyimpan hasil rekaman pada memori lokal. Karena data video yang besar, masih sedikit dashcam yang dapat dipantau secara langsung pada \textit{server}, sehingga dashcam tidak dapat dipantau secara realtime. Untuk itu diperlukan cara lain agar dapat memonitoring kendaraan dengan mengirimkan data-data yang dapat memberikan informasi tentang kendaraan. Untuk dapat mengetahui informasi kendaraan tersebut dapat dikembangkan fitur sistem \textit{monitoring} yang dapat melakukan penyimpanan data secara langsung pada \textit{server}. Sistem \textit{monitoring} tersebut berkomunikasi dengan \textit{Electronic Control Unit} (ECU) untuk mendapatkan data dari kendaraan tersebut. Salah satu cara untuk dapat berkomunikasi dengan ECU umumnya yaitu digunakan protokol komunikasi \textit{Controller Area Network} (CAN). Oleh karena itu dapat dikembangkan dashcam yang dapat menyimpan data pada \textit{server} dan berkomunikasi menggunakan protokol CAN secara \textit{realtime}.




